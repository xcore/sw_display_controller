% Generated by Sphinx.
\def\sphinxdocclass[english]{xmosmodern}
\documentclass[  collection]{xmosmodern}
\usepackage[utf8]{inputenc}
\DeclareUnicodeCharacter{00A0}{\nobreakspace}

\usepackage{longtable}



\title{Level meter module}
\date{November 12, 2013}
\author{}
\newcommand{\sphinxlogo}{}
\newcommand{\releasename}{Release}
\usepackage{xsphinx}
\usepackage{threeparttable}
\usepackage{fancyvrb}
\usepackage{indent}
\renewcommand\bfcode\textbf
\renewcommand\bf\textbf
\graphicspath{{./}{./images/}}
\makeindex

\newcommand\PYGZat{@}
\newcommand\PYGZlb{[}
\newcommand\PYGZrb{]}

\setlength{\emergencystretch}{8em}
\start

\maketitle
\pretoc
\phantomsection\label{index::doc}

%summary!

% NON-FULLWIDTH SECTION



% NON-FULLWIDTH SECTION
\clearpage
\chapter{Overview}
\label{overview:level-meter-module}\label{overview::doc}\label{overview:overview}%summary!
\begin{inthisdocument}
\item \nameref{overview:features}
\item \nameref{overview:memory-requirements}
\item \nameref{overview:resource-requirements}
\end{inthisdocument}



The level meter module is used to create a level meter display of a data array on LCD. The rendered image is stored in SDRAM.



% NON-FULLWIDTH SECTION
\section{Features}
\label{overview:features}\begin{itemize}
\item   Non-blocking SDRAM management.

\item   Real time rendering.

\item   Color selection for the display.

\item   No real time constraints on the application.

\end{itemize}




% NON-FULLWIDTH SECTION
\section{Memory requirements}
\label{overview:memory-requirements}
\begin{tabular}{ll}
\Toprule
\textbf{Resource} & \textbf{Usage}\\
\midrule
Data & 1520 bytes\\
Program & 2160 bytes\\
\bottomrule
\end{tabular}




% NON-FULLWIDTH SECTION
\section{Resource requirements}
\label{overview:resource-requirements}
\begin{tabular}{ll}
\Toprule
\textbf{Resource} & \textbf{Usage}\\
\midrule
Channels & 1\\
Timers & 0\\
Clocks & 0\\
Cores & 1\\
\bottomrule
\end{tabular}




% NON-FULLWIDTH SECTION
\clearpage
\chapter{Hardware requirements}
\label{hw:sec-hardware}\label{hw::doc}\label{hw:recommended-hardware}%summary!


This module may be evaluated using the sliceKIT Modular Development Platform, available from digikey. Required board SKUs are:
\begin{itemize}
\item   XP-SKC-U16 (slicekit U16 Core Board) plus xTAG-2

\item   XA-SK-SCR480 (which includes a 480x272 color touch screen)

\item   XA-SK-SDRAM

\end{itemize}

\phantomsection\label{api:sec-lever-meter-api}

To build a project including the \verb`module_level_meter`
the following modules are required:
\begin{itemize}
\item   \verb`module_display_controller`

\item   \verb`module_sdram` in \verb`sc_sdram_burst` which handles the SDRAM

\item   \verb`module_lcd` in \verb`sc_lcd` which handles the LCD

\end{itemize}



The section below details the configuration defines and the APIs used in the application.



% NON-FULLWIDTH SECTION
\clearpage
\chapter{API}
\label{api:configuration-defines}\label{api::doc}%summary!


The color palette to be used for the level meter display can be
configured via the header \verb`level_meter_conf.h`. The defines are:
\begin{description}
\item[\textbf{LEVEL\_METER\_NCOLORS}]

This defines the number of color used for the level meter display.

\item[\textbf{LEVEL\_METER\_COLORS}]

This gives the colors used. The colors can be picked from the list given in \verb`level_meter.h`.

\end{description}




% FULLWIDTH SECTION (with summary)
\clearpage
\chapter{API}
\label{api:api}%summary!
\begin{fullwidth} % chapter!


The \verb`module_level_meter` functionality is defined in
\begin{itemize}
\item   \verb`level_meter.xc`

\item   \verb`level_meter.h`

\end{itemize}



The level\_meter API is:
\index{level\_meter (C function)}
\vspace{-1.5\baselineskip}

\texttt{\begin{tabbing}\phantomsection\label{api:level_meter}void level\_meter(\= chanend c\_dc,\\ 
\> unsigned frBufNo,\\ 
\> unsigned data{[}{]},\\ 
\> unsigned N,\\ 
\> unsigned maxData)
\end{tabbing}}

\vspace{-2mm}
\sloppy
\begin{indentation*}{\blockindentlen}{0mm}

This function renders the level meter display of a given data sequence.


It connects to display controller for storing the rendered image frame.


This function has the following parameters:

\begin{option}

\item[c\_dc]channel connecting display controller.
\item[frBufNo]index of frame buffer to be updated.
\item[data]array containing magnitude spectrum values.
\item[N]number of data values to be displayed as bars.
\item[maxData]Maximum possible data value
\end{option}
\vspace{-3mm}


\end{indentation*}
\fussy
\end{fullwidth}%


% NON-FULLWIDTH SECTION
\clearpage
\chapter{Programming guide}
\label{programming:}\label{programming:programming-guide}\label{programming::doc}%summary!
\begin{inthisdocument}
\item \nameref{programming:includes-and-configuration}
\item \nameref{programming:programming}
\item \nameref{programming:software-requirements}
\end{inthisdocument}



This section provides information on how to create an application using \verb`level_meter` API.



% NON-FULLWIDTH SECTION
\section{Includes and configuration}
\label{programming:includes-and-configuration}

The application needs to include \verb`level_meter.h`. A configuration file \verb`level_meter_conf.h` should be placed in the application directory. The color palette for the level meter display is defined in the configuration file.



% NON-FULLWIDTH SECTION
\section{Programming}
\label{programming:programming}

The level meter module uses the APIs of display controller module. A simple application function that uses \verb`level_meter` API is given below.

\vspace{-5pt}\begin{minipage}{\linewidth}
\begin{lstlisting}[basicstyle=\ttfamily\Smaller,resetmargins=true]
void app(c_dc, data, N)
{
  unsigned frBuf;

  // Create frame buffer
  frBuf = display_controller_register_image(c_dc, LCD_ROW_WORDS, LCD_HEIGHT);

  // Render level meter display frame and commit
  level_meter(c_dc, frBuf, data, N);
  display_controller_frame_buffer_commit(c_dc, frBuf);
}
\end{lstlisting}
\end{minipage}



\verb`c_dc` is the channel connecting display controller. \verb`data` is the array of unsigned data values to be displayed. \verb`N` is the number of data values.



% NON-FULLWIDTH SECTION
\section{Software requirements}
\label{programming:software-requirements}

The module is built on xTIMEcomposer version 13
The module can be used in version 13 or any higher version of xTIMEcomposer.



% NON-FULLWIDTH SECTION
\clearpage
\chapter{Example applications}
\label{examples::doc}\label{examples:example-applications}%last summary
\begin{inthisdocument}
\item \nameref{examples:app-display-spectrum}
\item \nameref{examples:app-display-spectrum-from-adc}
\end{inthisdocument}



This tutorial describes the demo applications that uses the level meter module. Section \Sec~\ref{hw:sec-hardware} describes the required hardware setup to run the demos.



% NON-FULLWIDTH SECTION
\section{app\_display\_spectrum}
\label{examples:app-display-spectrum}

This application uses display controller and other modules to create a level-meter kind of spectral display on an LCD for a simulated signal. This application demonstrates real-time rendering and display of spectrum by taking short-time fourier transform.



% NON-FULLWIDTH SECTION
\subsection{Getting Started}
\label{examples:getting-started}\begin{enumerate}
\item   Connect XA-SK-SDRAM Slice Card to the XP-SKC-U16 Slicekit Core board using the connector marked with \verb`SQUARE`.

\item   Connect XA-SK-SCR480 Slice Card with LCD to the XP-SKC-U16 Slicekit Core board using the connector marked with \verb`DIAMOND`.

\item   Select \verb`app_display_spectrum`. Build the project and run.

\end{enumerate}



The spectra of segments of mixed signal of two simulated chirp waveforms are displayed on LCD.



% NON-FULLWIDTH SECTION
\section{app\_display\_spectrum\_from\_adc}
\label{examples:app-display-spectrum-from-adc}

This application uses \verb`module_usb_tile_support` along with display controller and other modules to create a level-meter kind of spectral display on an LCD for an analog audio input. This application showcases the use of multichannel ADC in an xCORE-USB series XMOS device to sample the analog input and real-time rendering and display of spectrum by taking short-time fourier transform.



% NON-FULLWIDTH SECTION
\subsection{Getting Started}
\label{examples:id1}\begin{enumerate}
\item   Connect XA-SK-SDRAM Slice Card to the XP-SKC-U16 Slicekit Core board using the connector marked with \verb`SQUARE`.

\item   Connect XA-SK-SCR480 Slice Card with LCD to the XP-SKC-U16 Slicekit Core board using the connector marked with \verb`DIAMOND`.

\item   Connect XA-SK-MIXED SIGNAL Slice Card to the XP-SKC-U16 Slicekit Core board using the connector marked with \verb`A`.

\item   Give the two channels of audio input from a PC or a mobile to pins 1 and 2 of J2 on the mixed signal slice card using a suitable cable. The ground is connected to pin 4 of J3.

\item   Select \verb`app_display_spectrum_from_adc`. Build the project and run.

\item   Play an audio in the PC or mobile.

\end{enumerate}



The spectra of segments of mixed signal of two audio channels are displayed on LCD.



\posttoc

\finish
